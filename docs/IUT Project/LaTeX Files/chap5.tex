% Chapter 5
\chapter{بحث}

در فصل قبل، نتایج جامعی از عملکرد مدل پیشنهادی در سناریوهای مختلف ارائه شد. این نتایج نشان دادند که مدل پیشنهادی، با ترکیب هوشمندانه ویژگی‌های مختلف دارویی و استفاده از معماری عمیق چندوجهی، توانسته است عملکرد قابل قبولی در پیش‌بینی تداخلات دارویی داشته باشد. با این حال، تفسیر عمیق‌تر این نتایج و بررسی چالش‌های موجود می‌تواند به درک بهتر قابلیت‌ها و محدودیت‌های مدل کمک کند.

در این فصل، ابتدا به تفسیر نتایج از منظر کاربرد عملی در محیط‌های بالینی می‌پردازیم و اهمیت یافته‌های پژوهش را در زمینه‌های مختلف از جمله کشف دارو و توسعه داروهای جدید بررسی می‌کنیم. سپس، چالش‌های اساسی در پیش‌بینی تداخلات دارویی را مورد بحث قرار می‌دهیم و راهکارهایی برای غلبه بر این چالش‌ها در پژوهش‌های آینده پیشنهاد می‌کنیم.

هدف این فصل، ارائه دیدگاهی جامع‌تر نسبت به نتایج به دست آمده و بررسی پیامدهای آن‌ها برای آینده پژوهش در حوزه پیش‌بینی تداخلات دارویی است. همچنین، این فصل تلاش می‌کند تا مسیرهای جدید پژوهشی که از نتایج این مطالعه قابل استخراج هستند را شناسایی و معرفی کند.
