% References
\renewcommand{\bibname}{مراجع}
\addcontentsline{toc}{section}{مراجع}

\begin{thebibliography}{99}

\begin{latin}

\baselineskip=.7cm

% 1
\bibitem{ref_li2023}
Z. Li, Sh. Zhu, B. Shao, “DSN-DDI: an accurate and generalized framework for drug–drug interaction prediction by dual-view representation learning”, \textit{Briefings in Bioinformatics}, Jan. 2023.

% 2
\bibitem{ref_mostafapour2019}
V. Mostafapour, O. Dikenelli “Attention-Wrapped Hierarchical BLSTMs for DDI Extraction”, 2019.

% 3
\bibitem{ref_nyamabo2021}
A. Nyamabo, H. Yu, J. Shi, “SSI–DDI: substructure–substructure interactions for drug–drug interaction prediction”, \textit{Briefings in Bioinformatics}, Nov. 2021.

% 4
\bibitem{ref_he2023}
H. He, G. Chen, C. Yu-Chian, “3DGT-DDI: 3D graph and text based neural network for drug–drug interaction prediction”, \textit{Briefings in Bioinformatics},  May. 2023.

% 5
\bibitem{ref_asfand2024}
M. Asfand-e-yar, Q.Hashir, “Multimodal CNN-DDI: using multimodal CNN for drug to drug interaction associated events”, \textit{Scientific Reports}, Feb. 2024.

% 6
\bibitem{ref_wang2017}
W. Wang, X. Yang “Dependency-based long short term memory network for drug-drug interaction extraction”, \textit{BMC Bioinformatics}, Dec. 2017.

% 7
\bibitem{ref_wu2020}
H. Wu, Y. Xing “Drug-drug interaction extraction via hybrid neural networks on biomedical literature”, \textit{Journal of Biomedical Informatics}, Jun. 2020.

% 8
\bibitem{ref_lin2022}
S. Lin, W. Chen “MDDI-SCL: predicting multi-type drug-drug interactions via supervised contrastive learning”, \textit{Journal of Cheminformatics}, Nov. 2022.

% 9
\bibitem{ref_wang2024}
Y. Wang, Y. Xiong “DDIPrompt: Drug-Drug Interaction Event Prediction based on Graph Prompt Learning”, 2024.

% 10
\bibitem{ref_hines2011}
L. Hines, J. Murphy “Potentially Harmful Drug–Drug Interactions in the Elderly: A Review”, Dec. 2011.

% 11
\bibitem{ref_yang2022}
Z. Yang, W. Zhong “Learning size-adaptive molecular substructures for explainable drug–drug interaction prediction by substructure-aware graph neural network”, 2022.

% 12
\bibitem{ref_dai2020}
X. Dai, C. Hu, B. Pei, "MLRDA: Multi-Level Representation learning with Double Attention for Drug-Drug Interaction Prediction", \textit{Bioinformatics}, 2020.

% 13
\bibitem{ref_kumari2024}
D. Kumari, "A Study on Drug Similarity Measures for Predicting Drug-Drug Interactions and Severity Using Machine Learning Techniques:", \textit{SCITEPRESS - Science and Technology Publications}, 2024.

% 14
\bibitem{ref_drugbank}
Wishart, D. S. et al. Drugbank 5.0: A major update to the drugbank database for 2018. Nucleic Acids Res. 46, D1074–D1082 (2018).

% 15
\bibitem{ref_kegg}
Kanehisa, M., Goto, S., Furumichi, M., Tanabe, M. Hirakawa, M. Kegg for representation and analysis of molecular networks involving diseases and drugs. Nucleic Acids Res. 38, D355–D360 (2010).

% 16
\bibitem{ref_ryu2018}
J. Ryu, H. Kim, S. Lee “Deep learning improves prediction of drug–drug and drug–food interactions”, \textit{Proceedings of the National Academy of Sciences}, May. 2018.

% 17
\bibitem{ref_deng2020}
Y. Deng, Y. Qiu, X. Xu, “META-DDIE: predicting drug–drug interaction events with few-shot learning”, \textit{Briefings in Bioinformatics}, Jan. 2022.

% 18
\bibitem{ref_shi2024}
Y. Shi, M.He, J. Chen, “SubGE-DDI: A new prediction model for drug-drug interaction established through biomedical texts and drug-pairs knowledge subgraph enhancement”, \textit{PLOS Computational Biology}, Apr. 2024.

% 19
\bibitem{ref_cascorbi2012}
I. Cascorbi, "Drug interactions – principles, examples and clinical consequences", \textit{Deutsches Ärzteblatt international}, 2012.

% 20
\bibitem{ref_huang2013}
J. Haung, C. Niu, C. Green, "Systematic Prediction of Pharmacodynamic Drug-Drug Interactions through Protein-Protein-Interaction Network", \textit{PLoS Computational Biology}, 2013.

% 21
\bibitem{ref_xu2019}
M. Xu, S. Liu, J. Yin, "GENN-DDI: Graph Energy Neural Network for Drug-Drug Interaction Prediction", \textit{IEEE Access}, 2019.

% 22
\bibitem{ref_weininger1988}
D. Weininger, "SMILES, a chemical language and information system. 1. Introduction to methodology and encoding rules", \textit{Journal of Chemical Information and Computer Sciences}, 1988.

% 23
\bibitem{ref_rdkit}
G. Landrum, "RDKit: Open-source cheminformatics", \url{https://www.rdkit.org}, 2006.

% 24
\bibitem{ref_devlin2018}
J. Devlin, M. W. Chang, K. Lee, K. Toutanova, "BERT: Pre-training of Deep Bidirectional Transformers for Language Understanding", \textit{arXiv preprint arXiv:1810.04805}, 2018.

% 25
\bibitem{ref_beltagy2019}
I. Beltagy, K. Lo, A. Cohan, "SciBERT: A Pretrained Language Model for Scientific Text", \textit{arXiv preprint arXiv:1903.10676}, 2019.

% 26
\bibitem{ref_glintborg2005}
B. Glintborg, S. Andersen, K. Dalhoff, "Drug-drug interactions among recently hospitalised patients – frequent but mostly clinically insignificant", \textit{European Journal of Clinical Pharmacology}, 2005.

% 27
\bibitem{ref_sokolova2009}
M. Sokolova, G. Lapalme, “A systematic analysis of performance measures for classification tasks”, \textit{Information Processing \& Management}, 2009.

% 28
\bibitem{ref_powers2011}
D. Powers, “Evaluation: From precision, recall and F-measure to ROC, informedness, markedness and correlation”, 2011.

% 29
\bibitem{ref_manning2008}
C. Manning, P. Raphavan, H. Schütze, “Introduction to information retrieval”, \textit{Cambridge university press}, 2008.

% 30
\bibitem{ref_github}
\textit{DrugVisualization}, \url{https://github.com/SayedHamidTabatabaei/DrugVisualization}, Dec. 2024.

% \bibitem{ref_alrabeah2022}
% M. Al Rabeah, A. Lakizadeh “GNN-DDI: A New Data Integration Framework for Predicting Drug-Drug Interaction Events Based on Graph Neural Networks”, \textit{Research Square}, Jul. 2022.


% \bibitem{ref_bradley1997}
% A. Bradley, “The use of the area under the ROC curve in the evaluation of machine learning algorithms”, \textit{Pattern Recognition}, Jul. 1997.


% \bibitem{ref_hand2001}
% D. Hand, R. Till, “A simple generalization of the area under the ROC curve for multiple class classification problems”, \textit{Machine Learning}, 2001.


% \bibitem{ref_davis2006}
% J. Davis, M. Goadrich, “The relationship between Precision-Recall and ROC curves”, \textit{ACM Press}, 2006.


% \bibitem{ref_saito2015}
% T. Saito, M. Rehmsmeier, “The precision-recall plot is more informative than the ROC plot when evaluating binary classifiers on imbalanced datasets”, \textit{PLOS ONE}, 2015.


\end{latin}


\end{thebibliography}