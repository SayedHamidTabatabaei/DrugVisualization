% Chapter 5
\chapter{نتیجه‌گیری و پیشنهادات برای کارهای آینده}

در این پژوهش، یک مدل یادگیری عمیق جدید برای پیش‌بینی و دسته‌بندی تداخلات دارویی معرفی شد که به‌واسطه ترکیب انواع مختلف داده‌های دارویی شامل اطلاعات ساختاری، شباهت دارویی و داده‌های متنی، یک رویکرد چندوجهی برای حل این مسئله ارائه می‌دهد. آزمایش‌های انجام‌شده نشان می‌دهند که مدل پیشنهادی توانایی بالایی در پیش‌بینی دقیق انواع تداخلات دارویی دارد. در ادامه، این فصل به بررسی یافته‌های اصلی پژوهش و نتایج کلیدی حاصل از آن می‌پردازد و سپس دستاوردها و نوآوری‌های مدل معرفی‌شده تشریح می‌شوند. همچنین، پیشنهاداتی برای ادامه تحقیقات و گسترش این مدل در کارهای آینده ارائه می‌شود تا مسیر تحقیقاتی پیش رو روشن‌تر گردد.

\section{خلاصه یافته‌ها}

در این پژوهش، یک مدل یادگیری عمیق برای پیش‌بینی تداخلات دارویی ارائه شد که به چندین یافته کلیدی دست یافت. در ارزیابی عملکرد مدل، نتایج نشان داد که در سناریوی اعتبارسنجی متقاطع، مدل با دقت \lr{96.5\%} توانست الگوهای تداخلات دارویی را شناسایی کند. هرچند در سناریوی دوم دقت مدل به \lr{70.9\%} کاهش یافت، ولی مدل توانست تداخلات دارویی را برای داروهای جدید پیش‌بینی کند. در نهایت، در سخت‌ترین سناریو (سناریوی سوم)، مدل با دقت \lr{51.1\%} عملکرد قابل قبولی داشت که نشان‌دهنده توانایی آن در پیش‌بینی تداخلات با داروهای کاملاً جدید است.	

یکی دیگر از یافته‌های مهم، اهمیت نسبی ویژگی‌های مختلف در مدل بود. به‌طور خاص، شبکه عصبی گرافی توانست در پردازش ساختار مولکولی داروها عملکرد برجسته‌ای داشته باشد، به‌ویژه در سناریوی اول که این ویژگی به طرز قابل توجهی دقت مدل را افزایش داد. ویژگی‌های مسیرهای بیولوژیکی نیز نقشی کلیدی در بهبود دقت پیش‌بینی‌ها ایفا کردند. علاوه بر این، ویژگی‌های متنی، به‌ویژه توصیف طبقه‌بندی داروها (\lr{CD})، در سناریوهای مربوط به داروهای جدید از اهمیت بیشتری برخوردار بودند. مقادیر Macro و Weighted در سناریوهای مختلف نزدیک به مقادیر Micro بود که نشان‌دهنده عملکرد متوازن مدل است.

از دیگر نتایج مهم، قابلیت تعمیم مدل به داروهای جدید بود. در حالی که مدل در پیش‌بینی تداخلات برای داروهای شناخته‌شده عملکرد بسیار خوبی نشان داد، کاهش عملکرد در سناریوهای دوم و سوم نشان داد که پیش‌بینی تداخلات برای داروهای جدید همچنان چالش‌برانگیز است. با این حال، ترکیب ویژگی‌های مختلف به بهبود قابلیت تعمیم مدل کمک کرده است.

در نهایت، معماری چندوجهی مدل مزایای قابل توجهی داشت. استفاده همزمان از اطلاعات ساختاری، شباهت دارویی و داده‌های متنی به درک جامع‌تری از تداخلات دارویی کمک کرد و اشتراک‌گذاری پارامترها در بخش‌های مختلف مدل موجب کاهش پیچیدگی و بهبود کارایی آن شد. معماری ماژولار مدل نیز امکان بهبود و تغییر بخش‌های مختلف آن را فراهم می‌کند.

\section{دستاوردها}

پژوهش حاضر دستاوردهای مهمی در زمینه پیش‌بینی تداخلات دارویی به همراه داشته است که در ابعاد نظری، تجربی، روش‌شناختی و عملی قابل بررسی است. یکی از دستاوردهای برجسته این پژوهش، ارائه معماری جدیدی برای مدل‌های یادگیری عمیق بود که به‌طور همزمان از سه نوع داده دارویی استفاده می‌کند \cite{ref_li2023}. این مدل با ترکیب شبکه‌های عصبی گرافی، شبکه‌های عصبی معمولی و مدل‌های زبانی، یک رویکرد جامع برای پیش‌بینی تداخلات دارویی ارائه می‌دهد. علاوه بر این، در پردازش ساختار مولکولی داروها از شبکه توجه گراف استفاده شد که قابلیت استخراج ویژگی‌های پیچیده از ساختار مولکولی را به مدل می‌دهد و دقت پیش‌بینی‌ها را به‌طور قابل‌توجهی افزایش می‌دهد \cite{ref_nyamabo2021}. همچنین، در این پژوهش، روش جدیدی برای اشتراک‌گذاری پارامترها میان بخش‌های مختلف مدل معرفی شده است که علاوه بر کاهش پیچیدگی محاسباتی، به بهبود عملکرد مدل نیز کمک کرده است \cite{ref_dai2020}.

در بعد تجربی، این پژوهش به بهبود دقت پیش‌بینی تداخلات دارویی در مقایسه با روش‌های موجود دست یافته است. به‌ویژه در سناریوی اول، مدل پیشنهادی توانست دقت بالایی در پیش‌بینی تداخلات دارویی از خود نشان دهد \cite{ref_he2023}. برای ارزیابی عملکرد مدل، سه سناریو مختلف طراحی شد که امکان بررسی قابلیت‌های مدل در شرایط مختلف را فراهم آورد \cite{ref_deng2020}. مدل پیشنهادی همچنین قادر به پیش‌بینی تداخلات دارویی حتی برای داروهای جدید است؛ اگرچه دقت پیش‌بینی در این حالت کمتر از داروهای شناخته‌شده بود، این امر نشان‌دهنده پتانسیل کاربردی مدل در دنیای واقعی است \cite{ref_shi2024}.

از منظر روش‌شناختی، این پژوهش روش‌های جدیدی برای ترکیب ویژگی‌های مختلف دارویی ارائه کرده است. این رویکرد می‌تواند به عنوان یک مدل برای مسائل مشابه در حوزه یادگیری ماشین استفاده شود و به چالش‌های پیشین در این زمینه پاسخ دهد \cite{ref_lin2022}. علاوه بر این، روش‌های مدل‌سازی کارآمدتری برای داده‌های دارویی توسعه یافته است که شامل تبدیل ساختار مولکولی به گراف و استخراج ویژگی‌های متنی می‌باشد \cite{ref_weininger1988}. همچنین، با استفاده از معیارهای ارزیابی جامع، امکان ارزیابی دقیق‌تر عملکرد مدل فراهم شده است.

در بعد عملی، این پژوهش در ارتقاء ایمنی دارویی کمک کرده است. ابزاری که در این تحقیق توسعه داده شده می‌تواند به پزشکان در شناسایی تداخلات دارویی خطرناک کمک کند و به این ترتیب ایمنی تجویز داروها را افزایش دهد \cite{ref_cascorbi2012}. همچنین، روش‌های ارائه شده می‌توانند در مراحل اولیه کشف داروهای جدید برای پیش‌بینی تداخلات احتمالی استفاده شوند و به تسریع فرآیند کشف دارو کمک کنند \cite{ref_ryu2018}. این پژوهش همچنین در کاهش هزینه‌های درمانی موثر است، زیرا می‌تواند از بروز عوارض جانبی ناشی از تداخلات دارویی جلوگیری کرده و در نتیجه هزینه‌های درمانی را کاهش دهد \cite{ref_huang2013}.

\section{پیشنهادات برای تحقیقات آینده}

با توجه به نتایج به دست آمده و مشاهدات انجام شده در طول این پژوهش، پیشنهادهای زیر برای ادامه این مسیر تحقیقاتی ارائه می‌شوند:

\subsection{بهبود پردازش داده‌های متنی}
یکی از مسیرهای مهم برای بهبود مدل، استفاده از مدل‌های زبانی پیشرفته‌تر به جای \lr{SciBERT} است \cite{ref_beltagy2019}. مدل‌های زبانی مانند \lr{PubMedBERT} که به طور خاص بر روی متون پزشکی و دارویی آموزش دیده‌اند، می‌توانند درک عمیق‌تری از متون تخصصی ارائه دهند \cite{ref_devlin2018}. همچنین مدل \lr{BioBERT} که بر روی مقالات زیست‌پزشکی آموزش دیده است، می‌تواند در استخراج روابط پیچیده بین داروها و سازوکار‌های عملکرد آن‌ها مؤثرتر عمل کند \cite{ref_he2023}.

علاوه بر این، استفاده از معماری \lr{Multi-Head Attention} برای پردازش داده‌های متنی می‌تواند به بهبود عملکرد مدل کمک کند \cite{ref_dai2020}. این معماری با توجه به توانایی در یادگیری روابط وابستگی در سطوح مختلف، می‌تواند جنبه‌های مختلف اطلاعات متنی را به طور همزمان پردازش کند. برای مثال، می‌تواند ارتباط بین عوارض جانبی، سازوکار عمل و موارد مصرف داروها را به طور همزمان در نظر بگیرد و وزن‌های متفاوتی به هر یک از این جنبه‌ها اختصاص دهد \cite{ref_shi2024}. این رویکرد می‌تواند به درک بهتر پیچیدگی‌های موجود در تداخلات دارویی کمک کند.

همچنین، امکان استفاده از ترکیب چندین مدل زبانی وجود دارد \cite{ref_wang2024}. این رویکرد می‌تواند از مزایای هر مدل بهره برده و نقاط ضعف آن‌ها را پوشش دهد. برای مثال، می‌توان از \lr{SciBERT} برای درک متون علمی عمومی، از \lr{PubMedBERT} برای متون پزشکی، و از \lr{BioBERT} برای متون مرتبط با سازوکار‌های مولکولی استفاده کرد و نتایج آن‌ها را به روشی هوشمندانه ترکیب نمود.

\subsection{بهینه‌سازی کاهش ابعاد}
جایگزینی شبکه‌های عصبی پرسپترون چندلایه با رمزنگارهای خودکار می‌تواند به کاهش ابعاد مؤثرتر داده‌ها منجر شود \cite{ref_he2023}. رمزنگارهای خودکار با یادگیری غیرخطی ویژگی‌ها، می‌توانند بازنمایی فشرده‌تر و معنادارتری از داده‌ها ارائه دهند. این روش در مطالعات اخیر در زمینه تداخلات دارویی نتایج امیدوارکننده‌ای نشان داده است \cite{ref_dai2020}. 

استفاده از رمزنگارهای خودکار در ترکیب با \lr{Multi-Head Attention} می‌تواند رویکرد جدیدی برای استخراج ویژگی‌های مهم از داده‌های با ابعاد بالا ارائه دهد \cite{ref_shi2024}. در این روش، رمزنگارهای خودکار می‌توانند ابعاد داده‌ها را کاهش دهند، در حالی که \lr{Multi-Head Attention} روابط مهم بین ویژگی‌های باقی‌مانده را حفظ می‌کند. این ترکیب به‌ویژه برای داده‌های شباهت دارویی که ماهیت چندبعدی دارند، می‌تواند مفید باشد \cite{ref_deng2020}.

با این حال، باید توجه داشت که استفاده از این روش‌ها نیازمند منابع محاسباتی قوی‌تر است. در مقایسه با شبکه‌های پرسپترون چندلایه، رمزنگارهای خودکار پارامترهای بیشتری برای آموزش دارند و زمان آموزش طولانی‌تری نیاز دارند \cite{ref_dai2020}. همچنین، تنظیم معماری رمزنگارهای خودکار شامل انتخاب تعداد مناسب لایه‌ها و واحدهای پنهان، نیازمند آزمایش‌های گسترده است تا بهترین ساختار برای حفظ اطلاعات مهم و حذف نویز به دست آید.

\subsection{استفاده از اطلاعات ساختار سه‌بعدی داروها}
یکی از یافته‌های جالب در این پژوهش، وجود فایل‌های \lr{SDF} در پایگاه داده \lr{DrugBank} است که اطلاعات ساختار دو‌بعدی و سه‌بعدی مولکول‌های دارویی را در خود دارند \cite{ref_drugbank}. هرچند این اطلاعات برای تمامی داروها موجود نیست، اما می‌تواند منبع ارزشمندی برای بهبود پیش‌بینی تداخلات دارویی باشد.

استفاده از شبکه‌های عصبی کانولوشنی (\lr{CNN}) برای پردازش این تصاویر می‌تواند اطلاعات ارزشمندی درباره ساختار فضایی داروها فراهم کند \cite{ref_asfand2024}. \lr{CNN}ها توانایی بالایی در استخراج ویژگی‌های مهم از داده‌های تصویری دارند و می‌توانند الگوهای فضایی مرتبط با تداخلات دارویی را شناسایی کنند. این رویکرد به‌ویژه در مطالعات اخیر که از ساختارهای سه‌بعدی برای پیش‌بینی تداخلات دارویی استفاده کرده‌اند، نتایج امیدوارکننده‌ای نشان داده است \cite{ref_he2023}.

ترکیب اطلاعات استخراج شده از ساختارهای سه‌بعدی با سایر ویژگی‌های دارویی می‌تواند به بهبود قابل توجه در دقت پیش‌بینی تداخلات دارویی منجر شود، زیرا بسیاری از تداخلات دارویی به ساختار فضایی مولکول‌ها وابسته هستند \cite{ref_yang2022}. برای مثال، شکل و ساختار فضایی مولکول‌ها می‌تواند در نحوه اتصال آن‌ها به گیرنده‌های مشترک و در نتیجه ایجاد تداخل تأثیرگذار باشد \cite{ref_weininger1988}.

\subsection{بهبود معماری کلی مدل}

ترکیب روش‌های پیشنهادی فوق می‌تواند به یک معماری جامع‌تر و قوی‌تر منجر شود. استفاده همزمان از رمزنگارهای خودکار برای کاهش ابعاد، \lr{Multi-Head Attention} برای پردازش متن و داده‌های شباهت، \lr{CNN} برای پردازش اطلاعات ساختاری سه‌بعدی، و مدل‌های زبانی پیشرفته‌تر می‌تواند یک مدل چندوجهی قدرتمند ایجاد کند \cite{ref_shi2024}. این معماری ترکیبی می‌تواند از مزایای هر روش بهره برده و محدودیت‌های آن‌ها را پوشش دهد \cite{ref_he2023}.

با این حال، پیاده‌سازی چنین معماری پیچیده‌ای با چالش‌هایی همراه است. یکی از مهم‌ترین چالش‌ها، نحوه ترکیب بهینه خروجی‌های بخش‌های مختلف مدل است \cite{ref_dai2020}. برای مثال، باید تصمیم گرفت که آیا خروجی‌ها باید به صورت موازی ترکیب شوند یا به صورت سلسله‌مراتبی پردازش شوند. همچنین، تعیین وزن مناسب برای هر بخش در ترکیب نهایی نیازمند آزمایش‌های گسترده است \cite{ref_yang2022}.

یک رویکرد امیدوارکننده برای حل این چالش‌ها، استفاده از سازوکار‌های توجه تطبیقی است که می‌توانند به طور خودکار اهمیت نسبی هر بخش را در زمان پیش‌بینی تعیین کنند \cite{ref_lin2022}. این سازوکار‌ها می‌توانند وزن‌های مختلفی را بر اساس نوع تداخل و ویژگی‌های خاص داروها به بخش‌های مختلف مدل اختصاص دهند.

علاوه بر این، می‌توان از تکنیک‌های یادگیری انتقالی\LTRfootnote{Transfer Learning} برای بهبود کارایی مدل در شرایط کمبود داده استفاده کرد \cite{ref_deng2022}. در این روش، مدل‌های مختلف ابتدا به صورت جداگانه روی داده‌های عمومی‌تر آموزش می‌بینند و سپس برای وظیفه خاص پیش‌بینی تداخلات دارویی تنظیم می‌شوند.

\vspace{1cm}

برای تسهیل پژوهش‌های آینده در این حوزه، کد کامل پیاده‌سازی مدل در \cite{ref_github} در دسترس قرار داده شده است. این پیاده‌سازی در قالب یک وب‌سایت طراحی شده و شامل سه بخش اصلی است: بخش رابط کاربری با استفاده از \lr{HTML}، \lr{CSS} و \lr{JavaScript}، بخش پردازشی با استفاده از \lr{Python} که شامل کدهای پیاده‌سازی و آموزش مدل است، و یک پایگاه داده \lr{MySQL} برای ذخیره‌سازی داده‌ها. مستندات کامل نحوه راه‌اندازی و اجرای پروژه به همراه تصاویری از محیط وب‌سایت در مخزن گیت‌هاب موجود است. با توجه به حجم بالای پایگاه داده این بخش به صورت جداگانه و از طریق لینک مشخص شده در مستندات قابل دریافت است.
