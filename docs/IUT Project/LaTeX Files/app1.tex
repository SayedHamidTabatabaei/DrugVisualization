
\appendixchapter{انواع تداخلات دارویی}
% \addcontentsline{toc}{chapter}{انواع تداخلات دارویی}
\label{appendix:drug-interactions}

فهرست زیر انواع مختلف تداخلات دارویی مورد بررسی در این پژوهش را نشان می‌دهد که از کدهای منتشر شده در مقاله \cite{ref_ryu2018} استخراج شده است. در تمامی موارد زیر، منظور از داروی اول و داروی دوم به ترتیب داروی تأثیرگذار و داروی تأثیرپذیر در تداخل دارویی است.


\setlength{\arrayrulewidth}{0.8pt} % ضخامت خطوط جدول

{
	\small
	\renewcommand{\arraystretch}{3}
	\begin{longtable}
		{|>{\centering\arraybackslash}p{0.08\textwidth}|>{\raggedright\arraybackslash}p{0.75\textwidth}|}
		\caption{انواع تداخلات دارویی } 
		\label{table:drug_interactions} \\ 
		\hline
		\rowcolor{lightgray}
		\textbf{کد} & \textbf{شرح تداخل}
		\endfirsthead
		\hline
		\rowcolor{lightgray}
		\textbf{کد} & \textbf{شرح تداخل}
		\endhead
		\hline
		1 & داروی اول ممکن است اثرات ضدروان‌پریشی داروی دوم را افزایش دهد. \\
		\hline
		2 & داروی اول ممکن است اثرات کاهش‌دهنده فشار خون داروی دوم را افزایش دهد. \\
		\hline
		3 & داروی اول ممکن است اثرات مسدودکنندگی عصبی-عضلانی داروی دوم را افزایش دهد. \\
		\hline
		4 & داروی اول ممکن است حساسیت به نور ناشی از داروی دوم را تشدید کند. \\
		\hline
		5 & داروی اول ممکن است اثرات افزایش دهنده قند خون داروی دوم را تقویت کند. \\
		\hline
		6 & داروی اول ممکن است اثرات سرکوب‌کننده مغز استخوان داروی دوم را افزایش دهد. \\
		\hline
		7 & داروی اول ممکن است آسیب‌های کبدی ناشی از داروی دوم را افزایش دهد. \\
		\hline
		8 & داروی اول ممکن است سرعت دفع داروی دوم را افزایش دهد که می‌تواند منجر به کاهش سطح سرمی و احتمالاً کاهش اثربخشی شود. \\
		\hline
		9 & دسترس‌پذیری زیستی داروی دوم ممکن است در صورت مصرف همزمان با داروی اول کاهش یابد. \\
		\hline
		10 & متابولیسم داروی دوم ممکن است در صورت مصرف همزمان با داروی اول کاهش یابد. \\
		\hline
		11 & داروی اول ممکن است اثرات گشادکنندگی برونش‌های داروی دوم را کاهش دهد. \\
		\hline
		12 & غلظت سرمی داروی دوم ممکن است در صورت مصرف همزمان با داروی اول کاهش یابد. \\
		\hline
		13 & داروی اول ممکن است اثرات تحریک‌کنندگی عصبی داروی دوم را افزایش دهد. \\
		\hline
		14 & داروی اول ممکن است اثرات تضعیف‌کننده سیستم عصبی مرکزی داروی دوم را تشدید کند. \\
		\hline
		15 & داروی اول ممکن است اثرات زخم‌زایی داروی دوم را افزایش دهد. \\
		\hline
		16 & داروی اول ممکن است احتباس مایعات ناشی از داروی دوم را تشدید کند. \\
		\hline
		17 & خطر یا شدت طولانی شدن فاصله QTc می‌تواند در مصرف همزمان داروی اول با داروی دوم افزایش یابد. \\
		\hline
		18 & خطر یا شدت خونریزی ممکن است در صورت مصرف همزمان داروی اول با داروی دوم افزایش یابد. \\
		\hline
		19 & داروی اول ممکن است اثرات کاهنده فشار خون داروی دوم را تشدید کند. \\
		\hline
		20 & داروی اول ممکن است اثرات کاهنده قند خون داروی دوم را تقویت کند. \\
		\hline
		21 & خطر واکنش حساسیتی به داروی دوم در صورت مصرف همزمان با داروی اول افزایش می‌یابد. \\
		\hline
		22 & داروی اول ممکن است خطر میوپاتیک و رابدومیولیز داروی دوم را افزایش دهد. \\
		\hline
		23 & داروی اول ممکن است عوارض پوستی داروی دوم را تشدید کند. \\
		\hline
		24 & غلظت سرمی داروی دوم ممکن است در صورت مصرف همزمان با داروی اول افزایش یابد. \\
		\hline
		25 & خطر یا شدت فشار خون بالا می‌تواند در صورت مصرف همزمان داروی دوم با داروی اول افزایش یابد. \\
		\hline
		26 & داروی اول ممکن است اثرات طولانی‌کننده فاصله QTc داروی دوم را تشدید کند. \\
		\hline
		27 & داروی اول ممکن است اثرات تضعیف‌کننده تنفسی داروی دوم را افزایش دهد. \\
		\hline
		28 & داروی اول ممکن است اثرات افزایش‌دهنده پتاسیم خون داروی دوم را تشدید کند. \\
		\hline
		29 & داروی اول ممکن است اثرات ضد درد داروی دوم را کاهش دهد. \\
		\hline
		30 & داروی اول ممکن است افت فشار خون وضعیتی ناشی از داروی دوم را تشدید کند. \\
		\hline
		31 & داروی اول ممکن است آسیب‌های کلیوی ناشی از داروی دوم را افزایش دهد. \\
		\hline
		32 & داروی اول ممکن است اثرات گشادکنندگی عروقی داروی دوم را تقویت کند. \\
		\hline
		33 & اثربخشی درمانی داروی دوم در صورت استفاده همزمان با داروی اول ممکن است افزایش یابد. \\
		\hline
		34 & داروی اول ممکن است اثرات ترومبوژنیک داروی دوم را افزایش دهد. \\
		\hline
		35 & داروی اول ممکن است اثرات تنگ‌کنندگی برونش‌های داروی دوم را تشدید کند. \\
		\hline
		36 & داروی اول می‌تواند باعث افزایش جذب داروی دوم شود که منجر به افزایش غلظت سرمی و احتمالاً تشدید عوارض جانبی می‌شود. \\
		\hline
		37 & خطر یا شدت افزایش پتاسیم خون می‌تواند در صورت مصرف همزمان داروی اول با داروی دوم افزایش یابد. \\
		\hline
		38 & دسترس‌پذیری زیستی داروی دوم ممکن است در صورت مصرف همزمان با داروی اول افزایش یابد. \\
		\hline
		39 & غلظت سرمی متابولیت‌های فعال داروی دوم می‌تواند در صورت مصرف همزمان با داروی اول افزایش یابد. \\
		\hline
		40 & داروی اول ممکن است اثرات ضد انعقادی داروی دوم را کاهش دهد. \\
		\hline
		41 & داروی اول ممکن است اثرات سرکوب‌کننده سیستم ایمنی داروی دوم را تقویت کند. \\
		\hline
		42 & داروی اول ممکن است اثرات آرام‌بخشی داروی دوم را کاهش دهد. \\
		\hline
		43 & داروی اول ممکن است اثرات کاهنده فشار خون و تضعیف‌کننده سیستم عصبی مرکزی داروی دوم را تشدید کند. \\
		\hline
		44 & اتصال پروتئینی داروی دوم می‌تواند در صورت مصرف همزمان با داروی اول کاهش یابد. \\
		\hline
		45 & غلظت سرمی متابولیت‌های فعال داروی دوم می‌تواند در صورت مصرف همزمان با داروی اول کاهش یابد که منجر به کاهش اثربخشی می‌شود. \\
		\hline
		46 & داروی اول ممکن است سرعت دفع داروی دوم را کاهش دهد که می‌تواند منجر به افزایش غلظت سرمی آن شود. \\
		\hline
		47 & داروی اول ممکن است اثربخشی داروی دوم را به عنوان یک عامل تشخیصی کاهش دهد. \\
		\hline
		48 & داروی اول ممکن است اثرات مسدودکنندگی عصبی-عضلانی داروی دوم را کاهش دهد. \\
		\hline
		49 & داروی اول ممکن است اثرات تضعیف‌کننده سیستم عصبی مرکزی و افزاینده فشار خون داروی دوم را تشدید کند. \\
		\hline
		50 & داروی اول ممکن است سمیت عصبی مرکزی داروی دوم را افزایش دهد. \\
		\hline
		51 & داروی اول ممکن است اثرات افزایش‌دهنده ضربان قلب داروی دوم را تشدید کند. \\
		\hline
		52 & داروی اول ممکن است اثرات ضد پلاکتی داروی دوم را کاهش دهد. \\
		\hline
		53 & داروی اول ممکن است سمیت عصبی داروی دوم را افزایش دهد. \\
		\hline
		54 & داروی اول ممکن است اثرات مسدودکنندگی گره دهلیزی-بطنی داروی دوم را تشدید کند. \\
		\hline
		55 & داروی اول ممکن است سمیت قلبی داروی دوم را افزایش دهد. \\
		\hline
		56 & داروی اول ممکن است سمیت قلبی داروی دوم را کاهش دهد. \\
		\hline
		57 & داروی اول ممکن است اثرات ضد پلاکتی داروی دوم را افزایش دهد. \\
		\hline
		58 & داروی اول ممکن است اثرات تنگ‌کنندگی عروقی داروی دوم را تشدید کند. \\
		\hline
		59 & داروی اول ممکن است سمیت شنوایی داروی دوم را افزایش دهد. \\
		\hline
		60 & داروی اول ممکن است اثرات تحریک‌کنندگی داروی دوم را تشدید کند. \\
		\hline
		61 & اثربخشی درمانی داروی دوم در صورت استفاده همزمان با داروی اول ممکن است کاهش یابد. \\
		\hline
		62 & داروی اول ممکن است اثرات کاهنده کلسیم خون داروی دوم را تشدید کند. \\
		\hline
		63 & داروی اول ممکن است اثرات یبوست‌زایی داروی دوم را افزایش دهد. \\
		\hline
		64 & داروی اول ممکن است اثرات آرام‌بخش داروی دوم را افزایش دهد. \\
		\hline
		65 & داروی اول ممکن است اثرات سروتونرژیک داروی دوم را تشدید کند. \\
		\hline
		
	\end{longtable}
}
