%*************************************************
% In this file the abstract is typeset.
% Make changes accordingly.
%*************************************************

\addcontentsline{toc}{section}{چکیده}
\newgeometry{left=2.5cm,right=3cm,top=3cm,bottom=2.5cm,includehead=false,headsep=1cm,footnotesep=.5cm}
\setcounter{page}{1}
\pagenumbering{arabic}						% شماره صفحات با عدد
\thispagestyle{empty}

~\vfill

\subsection*{چکیده}
\begin{small}
\baselineskip=0.7cm

تداخلات دارویی یکی از چالش‌های مهم در حوزه پزشکی و داروسازی است که می‌تواند منجر به عوارض جانبی جدی و حتی مرگ‌ومیر شود. پیش‌بینی دقیق این تداخلات، به خصوص برای داروهای جدید، همچنان یک مسئله چالش‌برانگیز است. در این پژوهش، یک مدل یادگیری عمیق چندوجهی برای پیش‌بینی و دسته‌بندی تداخلات دارویی ارائه شده است که با ترکیب هوشمندانه سه نوع داده دارویی شامل ساختار مولکولی، شباهت‌های دارویی و اطلاعات متنی طراحی شده است. در این مدل، شبکه عصبی گرافی برای پردازش ساختار مولکولی، معیار جاکارد برای محاسبه شباهت‌ها، و مدل \LR{SciBERT} برای استخراج اطلاعات متنی به کار گرفته شده است. ارزیابی مدل در قالب سه سناریوی مختلف با پیچیدگی‌های متفاوت انجام شده و نتایج نشان می‌دهند که مدل پیشنهادی به خصوص در شناسایی تداخلات برای داروهای شناخته شده، عملکرد بهتری نسبت به روش‌های موجود دارد. معماری چندوجهی این مدل امکان تعمیم‌پذیری به داروهای جدید را نیز فراهم می‌کند و می‌تواند در محیط‌های بالینی برای کمک به تصمیم‌گیری پزشکان و همچنین در فرآیند توسعه داروهای جدید مورد استفاده قرار گیرد. \\

\noindent\textbf{کلمات کلیدی:} تداخلات دارویی، یادگیری عمیق، شبکه عصبی گرافی، پردازش زبان طبیعی، داده‌کاوی، یادگیری چندوجهی، پیش‌بینی تداخلات، ایمنی دارویی
\end{small}