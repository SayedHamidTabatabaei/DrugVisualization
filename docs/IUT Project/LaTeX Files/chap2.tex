% Chapter 2
\chapter{مفاهیم پایه و پیشینه پژوهش}


فصل حاضر به بررسی پیشینه پژوهش در حوزه تداخلات دارویی اختصاص دارد. در ابتدا، مفاهیم بنیادین در زمینه تداخلات دارویی، شامل تعاریف، انواع، و سازوکار‌های اصلی این تداخلات، مورد بحث قرار می‌گیرد. آشنایی با این مفاهیم، درک پایه‌ای از پدیده تداخلات دارویی را برای خواننده فراهم می‌کند. بررسی پتانسیل آسیب‌رسانی این تداخلات، به‌ویژه در جمعیت سالمندان، اهمیت پژوهش در این حوزه را نشان می‌دهد \cite{ref_hines2011}. سپس، روش‌های موجود برای شناسایی تداخلات دارویی، از جمله روش‌های مبتنی بر قوانین، پایگاه‌های داده، و غربالگری، به طور کلی معرفی می‌شوند. بررسی نقاط قوت و محدودیت‌های هر یک از این روش‌ها، به درک بهتر چالش‌های موجود در این حوزه کمک می‌کند. در ادامه، کاربرد فناوری‌های نوین، به ویژه یادگیری ماشین، در مطالعات تداخلات دارویی مورد توجه قرار می‌گیرد. مروری بر مطالعات اخیر، پتانسیل این فناوری‌ها را در بهبود شناسایی و پیش‌بینی تداخلات دارویی نشان می‌دهد \cite{ref_ryu2018, ref_li2023, ref_nyamabo2021}. با این حال، چالش‌ها و محدودیت‌های پیش رو در استفاده از این روش‌ها نیز نباید از نظر دور بماند. در مجموع، این فصل تلاش دارد تا با ارائه یک دورنمای جامع از وضعیت فعلی دانش در زمینه تداخلات دارویی، زمینه را برای درک بهتر اهمیت و جایگاه پژوهش حاضر فراهم کند.

\section{مفاهیم پایه تداخلات دارویی}

تداخلات دارویی زمانی رخ می‌دهند که اثرات یک دارو تحت تأثیر مصرف همزمان داروی دیگر قرار می‌گیرد. این تداخلات می‌توانند باعث تغییر در اثربخشی درمانی، افزایش عوارض جانبی، یا بروز واکنش‌های نامطلوب شوند. شناسایی و پیشگیری از تداخلات دارویی برای بهبود ایمنی بیمار و بهینه‌سازی نتایج درمانی، به‌ویژه در بیمارانی که چندین دارو مصرف می‌کنند، حیاتی است \cite{ref_hines2011}. 


از نظر سازوکار عملکرد، تداخلات دارویی را می‌توان به دو دسته کلی تقسیم کرد: تداخلات فارماکوکینتیک و تداخلات فارماکودینامیک. تداخلات فارماکوکینتیک زمانی رخ می‌دهند که یک دارو بر روند جذب، پخش، متابولیسم، یا دفع داروی دیگر اثر می‌گذارد. از سوی دیگر، تداخلات فارماکودینامیک زمانی اتفاق می‌افتند که دو دارو اثرات مشابه یا متضادی بر بدن دارند \cite{ref_ryu2018}. در سطح مولکولی، سازوکار‌های مسئول تداخلات دارویی بسیار متنوع هستند و می‌توانند شامل تغییر در عملکرد پروتئین‌ها، آنزیم‌ها، کانال‌های یونی، و مسیرهای انتقال پیام در سلول‌ها باشند \cite{ref_he2023}. برای درک بهتر موضوع، می‌توان به نمونه‌های رایج اشاره کرد. به عنوان مثال، تداخلات بین ایبوپروفن که یک مسکن و ضدالتهاب رایج است و لیتیوم که برای درمان افسردگی استفاده می‌شود، نمونه‌ای از تداخلات فارماکوکینتیک است که در آن ایبوپروفن با کاهش دفع کلیوی لیتیوم، سطح آن را در خون افزایش می‌دهد \cite{ref_drugs}. از سوی دیگر، تداخلات بین آسپرین و ایبوپروفن که هر دو به عنوان داروهای ضد التهاب و مسکن شناخته می‌شوند، نمونه‌ای از تداخلات فارماکودینامیک است، زیرا این دو دارو به طور همزمان بر روی مسیر التهابی تأثیر می‌گذارند و مصرف همزمان آن‌ها می‌تواند خطر عوارض گوارشی را افزایش دهد \cite{ref_drugs}.

تلاش‌های متعددی برای طبقه‌بندی تداخلات دارویی صورت گرفته است. در این پژوهش، یک طرح جامع برای طبقه‌بندی تداخلات دارویی در 65 دسته استفاده شده است. این طبقه‌بندی بر اساس کدهای منبع باز مرتبط با مقاله \cite{ref_ryu2018} استخراج شده است\LTRfootnote{\rl{کدهای مربوطه در [https://bitbucket.org/kaistsystemsbiology/deepddi/src/master/] در دسترس هستند.}} این طبقه‌بندی، طیف وسیعی از تداخلات را پوشش می‌دهد، از جمله تداخلات مربوط به:

\begin{itemize}
	\item تغییر در غلظت یا اثربخشی داروها
	\item افزایش خطر عوارض جانبی مانند خونریزی یا نارسایی کلیه
	\item تداخل با عملکرد درمانی داروها مانند آنتی‌بیوتیک‌ها یا داروهای ضد فشار خون
	\item افزایش یا کاهش اثرات دارویی مانند اثرات آرام‌بخشی یا تحریک‌کنندگی
\end{itemize}

(برای فهرست کامل 65 نوع تداخل دارویی به همراه توضیحات، به پیوست 1 مراجعه کنید)
این طبقه‌بندی تفصیلی، امکان شناسایی دقیق‌تر الگوهای تداخلات دارویی و توسعه ابزارهای پیش‌بینی کننده کارآمدتر را فراهم می‌کند.

\section{روش‌های موجود برای شناسایی تداخلات دارویی}

در طول دهه‌های گذشته، روش‌های مختلفی برای شناسایی تداخلات دارویی توسعه یافته‌اند که می‌توان آنها را در سه دسته اصلی طبقه‌بندی کرد. رویکرد سنتی که همچنان کاربرد گسترده‌ای دارد، استفاده از پایگاه‌های داده و منابع اطلاعاتی مبتنی بر دانش است \cite{ref_ryu2018}. این منابع که شامل کتاب‌های مرجع، راهنماهای بالینی و پایگاه‌های اطلاعاتی تخصصی می‌شوند، اطلاعات شناخته شده در مورد تداخلات دارویی را جمع‌آوری و سازماندهی می‌کنند. در میان این پایگاه‌های داده، \textbf{DrugBank} یکی از جامع‌ترین منابع اطلاعات دارویی است که شامل بیش از 14,000 دارو با جزئیات کامل از ویژگی‌های شیمیایی، ساختار مولکولی، سازوکار عمل، مسیرهای متابولیک و تداخلات دارویی است \cite{ref_drugbank}. علاوه بر آن، پایگاه داده \textbf{KEGG} اطلاعات ارزشمندی در مورد مسیرهای متابولیک و تداخلات دارویی در سطح مولکولی ارائه می‌دهد \cite{ref_kegg}. این پایگاه‌ها به طور منظم به‌روزرسانی می‌شوند و داده‌های آن‌ها توسط متخصصان حوزه دارویی تأیید می‌شود که این امر اعتبار اطلاعات موجود در آن‌ها را تضمین می‌کند. متخصصان مراقبت‌های بهداشتی با مراجعه به این منابع و بررسی مشخصات دارویی بیماران، می‌توانند تداخلات بالقوه را شناسایی کنند. با این حال، به‌روزرسانی و حفظ جامعیت این منابع یکی از چالش‌های اصلی این رویکرد محسوب می‌شود.

با پیشرفت فناوری، سیستم‌های مبتنی بر قوانین توسعه یافتند که تداخلات دارویی را بر اساس مجموعه‌ای از قواعد از پیش تعریف شده بررسی می‌کنند \cite{ref_wang2017}. در این سیستم‌ها، متخصصان دامنه قوانینی را بر اساس عواملی مانند خواص دارویی، سازوکار عمل و شرایط بیمار تدوین می‌کنند که می‌تواند برای شناسایی تداخلات بالقوه مورد استفاده قرار گیرد. به عنوان مثال، یکی از قوانین رایج در این سیستم‌ها مربوط به تداخل داروهای مهارکننده آنزیم \lr{CYP3A4} با سایر داروهاست. این قانون می‌تواند به این صورت تعریف شود: "اگر داروی A یک مهارکننده قوی \lr{CYP3A4} باشد و داروی B توسط این آنزیم متابولیزه شود، احتمال افزایش غلظت داروی B در خون و بروز عوارض جانبی وجود دارد" \cite{ref_cascorbi2012}. سیستم‌های مبتنی بر قوانین با استفاده از مجموعه‌ای از چنین قواعدی، می‌توانند هشدارهای لازم را در هنگام تجویز داروها صادر کنند. مزیت اصلی این رویکرد، امکان ادغام آن در سیستم‌های نسخه‌نویسی الکترونیکی و سیستم‌های پشتیبانی تصمیم بالینی است که می‌توانند در نقطه مراقبت، هشدارهای لازم را ارائه دهند. البته تعریف و نگهداری مجموعه قوانین جامع و به‌روز همچنان یک چالش اساسی در این رویکرد است.

رویکرد نوین‌تر در این حوزه، استفاده از روش‌های مبتنی بر شواهد است که با بهره‌گیری از اطلاعات جمع‌آوری شده در پایگاه‌های داده بالینی، به شناسایی الگوهای تداخلات دارویی می‌پردازند \cite{ref_he2023}. این روش‌ها با تحلیل داده‌های واقعی بیماران که از منابعی مانند نسخه‌های الکترونیکی و سوابق سلامت به دست می‌آیند، می‌توانند بینش‌های ارزشمندی در مورد تداخلات رایج، عوارض جانبی و پیامدهای بالینی ارائه دهند. پیشرفت در حوزه داده‌کاوی و یادگیری ماشین، امکان استخراج دانش از حجم عظیم داده‌ها و کشف تداخلات بالقوه جدید را فراهم کرده است، هرچند که کیفیت و جامعیت داده‌های بالینی می‌تواند چالش‌برانگیز باشد.

مقایسه این سه رویکرد نشان می‌دهد که هر یک دارای نقاط قوت و محدودیت‌های خاص خود هستند. رویکردهای مبتنی بر دانش از اعتبار بالایی برخوردارند اما ممکن است جامع و به‌روز نباشند. سیستم‌های مبتنی بر قوانین قابلیت ادغام در جریان کار بالینی را دارند اما نگهداری آنها دشوار است. روش‌های مبتنی بر شواهد می‌توانند الگوهای جدید را کشف کنند اما به کیفیت داده‌ها وابسته هستند. به نظر می‌رسد ترکیبی از این سه رویکرد می‌تواند راهبرد جامع‌تری برای شناسایی و پیش‌بینی تداخلات دارویی فراهم کند، به خصوص اگر با فناوری‌های نوین مانند هوش مصنوعی تقویت شود. در بخش‌های بعدی، کاربرد فناوری‌های نوظهور برای ارتقاء قابلیت‌های این روش‌ها مورد بحث قرار خواهد گرفت.

\section{کاربرد یادگیری ماشین در شناسایی تداخلات دارویی}

یادگیری ماشین در طول دهه گذشته به یکی از ابزارهای قدرتمند در حوزه شناسایی و پیش‌بینی تداخلات دارویی تبدیل شده است. این تحول از روش‌های ساده مبتنی بر قوانین آغاز شد و با پیشرفت فناوری و افزایش حجم داده‌های در دسترس، به سمت روش‌های پیچیده‌تر و کارآمدتر حرکت کرده است. امروزه، روش‌های یادگیری ماشین با بهره‌گیری از الگوریتم‌های پیشرفته و مجموعه داده‌های بزرگ، قادر به کشف الگوها و روابط پنهان بین داروها هستند و می‌توانند تداخلات بالقوه را با دقت بالایی پیش‌بینی کنند. تکامل این روش‌ها به چهار مسیر اصلی منتهی شده است: روش‌های مبتنی بر شبکه‌های عصبی که از قدرت پردازش موازی برای یادگیری ویژگی‌های پیچیده استفاده می‌کنند، روش‌های مبتنی بر گراف که ساختار ذاتی روابط دارویی را مدل می‌کنند، روش‌های مبتنی بر مدل‌های مولد که با تولید داده‌های مصنوعی به بهبود عملکرد سیستم‌های پیش‌بینی کمک می‌کنند، و روش‌های ترکیبی که اطلاعات ساختاری و متنی را برای دستیابی به پیش‌بینی‌های دقیق‌تر با هم ادغام می‌کنند. این بخش به بررسی و مقایسه هر یک از این رویکرد می‌پردازد و نمونه‌های برجسته هر یک را معرفی می‌کند.

\subsection{روش‌های مبتنی بر شبکه‌های عصبی}

شبکه‌های عصبی به دلیل توانایی یادگیری خودکار ویژگی‌ها و قابلیت مدل‌سازی روابط پیچیده غیرخطی، به یکی از محبوب‌ترین رویکردها در حوزه پیش‌بینی تداخلات دارویی تبدیل شده‌اند. این روش‌ها با پیشرفت معماری‌های یادگیری عمیق، از شبکه‌های ساده به سمت معماری‌های پیچیده‌تر مانند یادگیری چندوظیفه‌ای و شبکه‌های کانولوشنی چندگانه تکامل یافته‌اند. این تکامل امکان پردازش همزمان انواع مختلف داده‌های دارویی و استخراج ویژگی‌های سطح بالا را فراهم کرده است.

در \cite{ref_ryu2018}، یک روش به نام \lr{DeepDDI} معرفی شده است که از شبکه‌های عصبی عمیق برای پیش‌بینی تداخلات دارویی استفاده می‌کند. این روش، ویژگی‌های ساختاری داروها را با استفاده از روش‌های تعبیه گراف\LTRfootnote{Graph Embedding} به فضای برداری نگاشت می‌کند. سپس، جفت بردارهای دارویی به عنوان ورودی به یک شبکه عصبی عمیق داده می‌شوند تا احتمال وجود تداخل بین آنها پیش‌بینی شود. نتایج این مطالعه نشان داد که \lr{DeepDDI} قادر است تداخلات دارویی را با دقت بالایی شناسایی کند و عملکرد بهتری نسبت به روش‌های سنتی مبتنی بر قاعده یا مبتنی بر شباهت داشته باشد.

در \cite{ref_deng2020}، یک روش دیگر به نام \lr{DDIMDL} پیشنهاد شده است که از یادگیری چندوظیفه‌ای\LTRfootnote{Multi-task Learning} برای بهبود پیش‌بینی تداخلات دارویی استفاده می‌کند. در این روش، علاوه بر وظیفه اصلی پیش‌بینی وجود تداخل بین دو دارو، وظایف جانبی دیگری مانند پیش‌بینی عوارض جانبی و پیش‌بینی تعامل دارو-پروتئین نیز در نظر گرفته می‌شوند. با آموزش همزمان مدل بر روی این وظایف مختلف، \lr{DDIMDL} می‌تواند اطلاعات مفیدی را از وظایف جانبی استخراج کند و برای بهبود عملکرد پیش‌بینی تداخلات دارویی استفاده نماید. نتایج آزمایشات نشان داد که این رویکرد چندوظیفه‌ای منجر به بهبود قابل توجهی در دقت پیش‌بینی تداخلات دارویی می‌شود.

در \cite{ref_asfand2024}، یک روش به نام \lr{MCNN-DDI} معرفی شده است که از شبکه‌های عصبی کانولوشنی چندگانه\LTRfootnote{Multi-Modal CNN} برای پیش‌بینی تداخلات دارویی استفاده می‌کند. این مدل، ویژگی‌های مختلفی از داروها شامل اهداف، آنزیم‌ها، مسیرها و ساختار شیمیایی را به عنوان ورودی دریافت می‌کند. برای هر نوع ویژگی، یک شبکه عصبی کانولوشنی مجزا اختصاص داده شده است. شباهت‌های بین داروها با استفاده از معیار شباهت جاکارد محاسبه می‌شود و به عنوان ویژگی اضافی به مدل داده می‌شود. در نهایت، خروجی‌های حاصل از شبکه‌های عصبی مختلف ترکیب می‌شوند تا پیش‌بینی نهایی تداخلات دارویی به دست آید.

در \cite{ref_lin2022}، روشی به نام \lr{MDDI-SCL} ارائه شده است که از ترکیب ویژگی‌های دارویی در چند مرحله برای پیش‌بینی تداخلات دارویی چندگانه استفاده می‌کند. در ابتدا، ویژگی‌های مختلف داروها توسط یک رمزگذار\LTRfootnote{Encoder} به فضای نهفته با ابعاد کمتر نگاشت می‌شوند. سپس، بردارهای نهفته حاصل برای هر دارو با هم ترکیب شده و به یک شبکه عصبی برای پیش‌بینی نوع تداخل دارویی داده می‌شوند. این روش، با در نظر گرفتن چندین نوع ویژگی دارویی و ترکیب آنها در مراحل مختلف، قادر به شناسایی تداخلات دارویی چندگانه با دقت بالا است.

\subsection{روش‌های مبتنی بر گراف}

علاوه بر شبکه‌های عصبی، روش‌های مبتنی بر گراف نیز در حوزه تداخلات دارویی مورد استفاده قرار گرفته‌اند. این روش‌ها با تمرکز بر ماهیت ارتباطی تداخلات دارویی، راه‌حل طبیعی و قدرتمندی برای این مسئله ارائه می‌دهند. در این روش‌ها، داروها به عنوان گره‌های گراف و ارتباطات آنها به عنوان یال‌های گراف مدل می‌شوند که اجازه می‌دهد ساختار ذاتی و توپولوژی روابط دارویی به طور مستقیم در فرآیند یادگیری در نظر گرفته شود. پیشرفت‌های اخیر در حوزه یادگیری گراف، به خصوص معرفی سازوکار‌های توجه و شبکه‌های عصبی انرژی گراف، امکانات جدیدی را برای مدل‌سازی دقیق‌تر این روابط فراهم کرده است.

در \cite{ref_nyamabo2021}، یک روش به نام \lr{SSI-DDI} معرفی شده است که از گراف‌های توجه\LTRfootnote{Graph Attention Networks} برای مدل‌سازی ساختار شیمیایی داروها استفاده می‌کند. در این روش، چندین لایه گراف توجه به کار گرفته شده و خروجی آنها توسط یک سازوکار توجه مشترک\LTRfootnote{Co-attention} ترکیب می‌شود تا اهمیت نسبی برهمکنش‌های زیرساختی مولکولی در پیش‌بینی تداخلات دارویی در نظر گرفته شود. این رویکرد، با تمرکز بر روی ساختار شیمیایی داروها و استفاده از سازوکار توجه، قادر به شناسایی تداخلات دارویی با دقت بالایی است.


در \cite{ref_xu2019} یک مدل شبکه عصبی انرژی گراف به نام \lr{GENN-DDI}\LTRfootnote{Graph Energy Neural Network for Drug-Drug Interaction} برای پیش‌بینی تداخلات دارویی ارائه شده است. این روش، از اطلاعات ساختاری داروها برای ایجاد یک گراف دارویی استفاده می‌کند که در آن هر گره نشان‌دهنده یک دارو و هر یال بیانگر تشابه ساختاری بین دو دارو است. سپس، با استفاده از یک شبکه عصبی انرژی گراف، تعاملات بین گره‌های دارو مدل‌سازی می‌شوند. این شبکه عصبی، انرژی لازم برای فعال‌سازی هر یال را محاسبه می‌کند که نشان‌دهنده احتمال وجود تداخل بین دو داروی متصل به آن یال است. مدل \lr{GENN-DDI} با بهینه‌سازی یک تابع هدف انرژی، می‌تواند پیش‌بینی‌های دقیقی از تداخلات دارویی ارائه دهد. این رویکرد مبتنی بر گراف انرژی، روشی جدید و نوآورانه برای مدل‌سازی تداخلات دارویی محسوب می‌شود. با این حال، یکی از محدودیت‌های احتمالی این روش، عدم در نظر گرفتن اطلاعات نحوی و معنایی در ساختار مولکولی داروها است که در مقاله اصلی \cite{ref_xu2019} به آن اشاره شده است.


\subsection{روش‌های مبتنی بر مدل‌های مولد}

با توجه به محدودیت داده‌های موجود برای برخی از انواع تداخلات دارویی و عدم توازن در مجموعه داده‌ها، استفاده از مدل‌های مولد به عنوان رویکردی امیدوارکننده مورد توجه قرار گرفته است \cite{ref_yu2023}. این مدل‌ها با یادگیری توزیع داده‌های واقعی و تولید نمونه‌های مصنوعی با کیفیت بالا، می‌توانند به بهبود عملکرد سیستم‌های پیش‌بینی تداخلات دارویی کمک کنند. به‌ویژه، شبکه‌های مولد تخاصمی (\lr{GAN}\LTRfootnote{Generative Adversarial Network}) با توانایی یادگیری همزمان ویژگی‌های ساختاری و الگوهای تداخلی، نتایج امیدوارکننده‌ای در این حوزه نشان داده‌اند \cite{ref_zhang2024}. اخیراً، روش‌های مبتنی بر \lr{GAN} با ترکیب معماری‌های پیشرفته و مکانیزم‌های توجه توانسته‌اند دقت پیش‌بینی تداخلات دارویی را به طور قابل توجهی افزایش دهند \cite{ref_he2023}.

روش \lr{DGANDDI} \cite{ref_yu2023} یک معماری دوگانه مبتنی بر شبکه‌های مولد تخاصمی برای پیش‌بینی تداخلات دارویی ارائه می‌دهد. این معماری از دو شبکه مولد تخاصمی موازی تشکیل شده که یکی بر روی ویژگی‌های ساختاری و دیگری بر روی الگوهای برهم‌کنش بین داروها تمرکز می‌کند. در هر شبکه، یک مولد\LTRfootnote{Generator} به تولید داده‌های مصنوعی می‌پردازد، در حالی که یک تشخیص‌دهنده\LTRfootnote{Discriminator} وظیفه تمایز بین داده‌های واقعی و مصنوعی را بر عهده دارد. این دو شبکه موازی، الگوهای مختلف موجود در داده‌های دارویی را استخراج می‌کنند و در نهایت از طریق یک مکانیزم یکپارچه‌سازی با هم ترکیب می‌شوند. سپس، خروجی ترکیب شده وارد یک شبکه عصبی عمیق می‌شود تا پیش‌بینی نهایی نوع تداخل را انجام دهد. این معماری دوگانه با بهره‌گیری از مزایای یادگیری تخاصمی، توانسته است عملکرد بهتری نسبت به روش‌های پیشین در پیش‌بینی انواع تداخلات دارویی داشته باشد.

\lr{RCAN-DDI} \cite{ref_zhang2024} یک شبکه تخاصمی متقاطع با آگاهی از روابط را برای پیش‌بینی تداخلات دارویی معرفی می‌کند. در این معماری، دو شبکه تخاصمی به صورت موازی کار می‌کنند که با یکدیگر تعامل متقاطع دارند. در شبکه اول، مولد با دریافت ویژگی‌های توپولوژیکی به تولید ویژگی‌های ساختاری مصنوعی می‌پردازد و تشخیص‌دهنده متناظر آن، ویژگی‌های ساختاری واقعی را از مصنوعی تشخیص می‌دهد. همزمان در شبکه دوم، مولد با دریافت ویژگی‌های ساختاری به تولید ویژگی‌های توپولوژیکی مصنوعی می‌پردازد و تشخیص‌دهنده آن، ویژگی‌های توپولوژیکی واقعی را از مصنوعی تمایز می‌دهد. این دو شبکه، اطلاعات مختلف موجود در داده‌های دارویی را استخراج می‌کنند و در نهایت از طریق یک مکانیزم یکپارچه‌سازی با هم ترکیب می‌شوند. سپس، خروجی ترکیب شده وارد یک شبکه عصبی عمیق می‌شود تا پیش‌بینی نهایی نوع تداخل را انجام دهد. این رویکرد با بهره‌گیری از یادگیری تخاصمی متقاطع، توانسته است دقت پیش‌بینی را به طور قابل توجهی بهبود بخشد.

\subsection{روش‌های مبتنی بر ترکیب اطلاعات ساختاری و متنی}

نسل جدید روش‌های پیش‌بینی تداخلات دارویی بر اساس این بینش توسعه یافته‌اند که ترکیب اطلاعات مختلف می‌تواند به درک جامع‌تری از تداخلات منجر شود. این روش‌ها با ادغام داده‌های ساختاری مولکول‌ها و اطلاعات متنی موجود در متون علمی و پزشکی، از مزایای هر دو نوع داده بهره می‌برند. پیشرفت‌های اخیر در پردازش زبان طبیعی، به خصوص مدل‌های زبانی پیش‌آموزش‌دیده مانند \lr{BERT}\LTRfootnote{Bidirectional Encoder Representations from Transformers} و مشتقات آن، امکان استخراج اطلاعات ارزشمند از متون پزشکی را فراهم کرده و ترکیب آن با ویژگی‌های ساختاری به بهبود قابل توجه دقت پیش‌بینی‌ها منجر شده است.

در \cite{ref_he2023}، یک روش به نام \lr{3DGT-DDI} پیشنهاد شده است که از گراف‌های سه‌بعدی برای مدل‌سازی ساختار مولکولی داروها و از مدل زبانی \lr{SCIBERT} برای استخراج ویژگی‌های متنی از توضیحات تداخلات دارویی استفاده می‌کند. ابتدا، ساختار سه‌بعدی مولکول‌ها به یک شبکه گرافی سه‌بعدی داده می‌شود تا ویژگی‌های ساختاری استخراج شوند. سپس، با استفاده از \lr{SCIBERT}، بردارهای نهفته از اطلاعات متنی مرتبط با تداخلات دارویی استخراج می‌شوند. در نهایت، این دو نوع ویژگی با هم ترکیب شده و برای پیش‌بینی تداخلات دارویی استفاده می‌شوند.

مشابه این رویکرد، در \cite{ref_shi2024}، یک روش به نام \lr{SubGE-DDI} ارائه شده است که از ترکیب اطلاعات گرافی و متنی برای پیش‌بینی تداخلات دارویی استفاده می‌کند. در این روش، ابتدا یک گراف دانش با استفاده از پایگاه‌های داده زیست‌پزشکی ایجاد می‌شود و زیرگراف‌های مرتبط با جفت داروها استخراج می‌شوند. سپس، با استفاده از مدل زبانی \lr{PubMedBERT}، ویژگی‌های متنی از توضیحات تداخلات دارویی استخراج می‌شوند. در نهایت، ویژگی‌های گرافی و متنی با هم ترکیب شده و به یک شبکه عصبی عمیق برای پیش‌بینی نوع تداخل دارویی داده می‌شوند. یکی از چالش‌های احتمالی این رویکرد، کیفیت و جامعیت پایگاه‌های داده زیست‌پزشکی مورد استفاده برای ایجاد گراف دانش است. کیفیت پایین داده‌ها می‌تواند بر عملکرد مدل تأثیر منفی بگذارد.


در \cite{ref_dai2020} یک روش به نام \lr{MLRDA}\LTRfootnote{Multi-Level Representation learning with Double Attention} پیشنهاد شده است که از یادگیری بازنمایی سلسله‌مراتبی چندسطحی با استفاده از دو سازوکار توجه برای پیش‌بینی تداخلات دارویی استفاده می‌کند. این روش، اطلاعات تداخلی داروها را در سه سطح مولکولی، سلولی و درمانی بررسی می‌کند. ابتدا، در هر سطح، یک سازوکار توجه برای استخراج ویژگی‌های مهم از ساختار داده‌ها به کار گرفته می‌شود. سپس، یک سازوکار توجه سطح بالاتر برای ترکیب اطلاعات از سطوح مختلف و تولید یک بازنمایی جامع از تداخلات دارویی استفاده می‌شود. در نهایت، این بازنمایی برای پیش‌بینی احتمال تداخل بین دو دارو مورد استفاده قرار می‌گیرد. نتایج این تحقیق نشان داد که استفاده از اطلاعات چندسطحی و سازوکار توجه می‌تواند به بهبود عملکرد پیش‌بینی تداخلات دارویی کمک کند.

به طور کلی، روش‌های مبتنی بر ترکیب اطلاعات ساختاری و متنی پتانسیل بالایی در غلبه بر محدودیت‌های روش‌های مبتنی بر ساختار یا متن به تنهایی دارند. ترکیب این دو نوع اطلاعات می‌تواند منجر به پیش‌بینی‌های جامع‌تر و دقیق‌تر تداخلات دارویی شود. در همین راستا، پژوهش حاضر نیز با درک اهمیت این رویکرد ترکیبی، علاوه بر استفاده از اطلاعات ساختاری مولکول‌ها به صورت گراف و داده‌های متنی، از اطلاعات مربوط به شباهت ویژگی‌های دارویی نیز بهره می‌برد تا به یک مدل جامع‌تر دست یابد. جزئیات این روش ترکیبی در فصل‌های بعدی به تفصیل مورد بحث قرار خواهد گرفت.

این نمونه‌ها، تنها بخشی از پژوهش‌های انجام شده در زمینه کاربرد یادگیری ماشین برای شناسایی و پیش‌بینی تداخلات دارویی هستند. با توجه به پیشرفت‌های اخیر در این حوزه و ظهور روش‌های نوین مانند شبکه‌های عصبی گراف و مدل‌های زبانی، انتظار می‌رود که در آینده شاهد توسعه روش‌های دقیق‌تر و کارآمدتر برای مدیریت تداخلات دارویی باشیم. با این حال، برای ارزیابی و مقایسه دقیق روش‌های مختلف یادگیری ماشین در این حوزه، استفاده از مجموعه داده‌های استاندارد و معیارهای ارزیابی مناسب اهمیت زیادی دارد. این امر به شناسایی نقاط قوت و ضعف هر روش و هدایت تحقیقات آینده کمک می‌کند. همچنین، همکاری بین محققان حوزه‌های مختلف، مانند علوم دارویی، بیوانفورماتیک و یادگیری ماشین، برای توسعه روش‌های نوآورانه و کاربردی در زمینه شناسایی و پیش‌بینی تداخلات دارویی ضروری است.
\begin{table}[!t]
	\caption{مقایسه روش‌های مختلف یادگیری ماشین در شناسایی تداخلات دارویی}
	\label{table:2-1}
	\centering % برای وسط‌چین کردن جدول
	\renewcommand{\arraystretch}{2.5} 
	\newcolumntype{P}[1]{>{\setstretch{1.5}\raggedleft\arraybackslash}p{#1}}
	\resizebox{\textwidth}{!}{
		\begin{tabular}{|P{2.7cm}|P{3.8cm}|P{4.2cm}|P{4.8cm}|}
			\hline
			\textbf{روش} & \textbf{داده‌های ورودی} & \textbf{مزایا} & \textbf{محدودیت‌ها} \\
			\hline
			DeepDDI \cite{ref_ryu2018} & ساختار مولکولی و تعبیه گراف &دقت بالا در پیش‌بینی \newline یادگیری خودکار ویژگی‌ها &نیاز به داده‌های آموزشی زیاد \newline پیچیدگی محاسباتی بالا \\
			\hline
			DDIMDL \cite{ref_deng2020} & ویژگی‌های دارویی چندگانه و داده‌های تعاملی & یادگیری چندوظیفه‌ای \newline استفاده از اطلاعات جانبی & پیچیدگی پیاده‌سازی \newline حساسیت به کیفیت داده‌های ورودی \\
			\hline
			MCNN-DDI \cite{ref_asfand2024} & ساختار مولکولی، اهداف، آنزیم‌ها، مسیرها و شباهت‌های جاکارد &  پردازش همزمان چندین نوع داده \newline تخصیص CNN مجزا برای هر نوع داده & پیچیدگی محاسباتی بالا \newline نیاز به تنظیم پارامترهای متعدد \\
			\hline
			MDDI-SCL \cite{ref_lin2022} & ویژگی‌های مختلف دارویی & کاهش ابعاد هوشمند ویژگی‌ها \newline پیش‌بینی تداخلات چندگانه & وابستگی به کیفیت رمزگذار \newline پیچیدگی در ترکیب ویژگی‌ها \\
			\hline
			SSI-DDI \cite{ref_nyamabo2021} & ساختار شیمیایی و گراف‌های توجه & تمرکز بر زیرساختارهای مهم \newline قابلیت تفسیرپذیری بالا & محدودیت در مدل‌سازی تعاملات پیچیده \newline وابستگی به کیفیت داده‌های ساختاری \\
			\hline
			GENN-DDI \cite{ref_xu2019} & ساختار مولکولی و تشابه ساختاری & مدل‌سازی انرژی‌محور تعاملات \newline نوآوری در محاسبه احتمال تداخل & عدم استفاده از اطلاعات نحوی و معنایی \newline محدودیت در پیش‌بینی‌های پیچیده \\
			\hline
			DGANDDI \cite{ref_yu2023} & ساختار مولکولی و ویژگی‌های عملکردی & یادگیری همزمان ویژگی‌های ساختاری و عملکردی \newline کاهش مشکل عدم توازن داده‌ها & پیچیدگی محاسباتی بالا \newline نیاز به داده‌های آموزشی زیاد \\
			\hline
			RCAN-DDI \cite{ref_zhang2024} & ساختار مولکولی و روابط دارویی & یادگیری روابط پیچیده بین ویژگی‌ها \newline مکانیزم توجه رابطه‌ای & پیچیدگی پیاده‌سازی \newline حساسیت به تنظیم پارامترها \\
			\hline
			\lr{3DGT-DDI} \cite{ref_he2023} & ساختار سه‌بعدی و داده‌های متنی & ترکیب اطلاعات ساختاری و متنی \newline دقت بالا در پیش‌بینی & نیاز به ساختار سه‌بعدی دقیق \newline پیچیدگی محاسباتی بالا \\
			\hline
			SubGE-DDI \cite{ref_shi2024} & گراف دانش و متون زیست‌پزشکی & استفاده از دانش دامنه\newline قابلیت تعمیم‌پذیری بالا & وابستگی به کیفیت گراف دانش\newline نیاز به به‌روزرسانی مداوم \\
			\hline
			MLRDA \cite{ref_dai2020} & داده‌های چندسطحی (مولکولی، سلولی و درمانی) & تحلیل چندسطحی\newline سازوکار توجه دوگانه & پیچیدگی در جمع‌آوری داده‌های چندسطحی\newline نیاز به منابع محاسباتی زیاد \\
			\hline
		\end{tabular}
	}
	
\end{table}

جدول \ref{table:2-1} مقایسه‌ای بین روش‌های مختلف یادگیری ماشین که برای شناسایی تداخلات دارویی مورد استفاده قرار گرفته‌اند را نشان می‌دهد. این مقایسه بر اساس نوع داده‌های ورودی مورد نیاز، مزایای اصلی و محدودیت‌های ویژه هر روش انجام شده است. هدف از ارائه این مقایسه، کمک به درک بهتر نقاط قوت و ضعف هر رویکرد و شناسایی حوزه‌های مناسب برای کاربرد آن‌ها است.


\section{شکاف تحقیقاتی}

با وجود پیشرفت‌های قابل توجه در زمینه پیش‌بینی تداخلات دارویی با استفاده از روش‌های یادگیری ماشین، همچنان شکاف‌های مهمی در این حوزه وجود دارد \cite{ref_shi2024}. یکی از مهم‌ترین این شکاف‌ها، محدودیت در استفاده همزمان از منابع مختلف داده است. اکثر مطالعات موجود تنها بر یک یا دو نوع از داده‌های دارویی تمرکز کرده‌اند. برای مثال، برخی مطالعات فقط از ساختار مولکولی استفاده می‌کنند \cite{ref_nyamabo2021}، در حالی که برخی دیگر تنها بر داده‌های متنی تکیه دارند \cite{ref_he2023}. نیاز به رویکردی جامع که بتواند انواع مختلف داده‌های دارویی را به طور همزمان پردازش و ترکیب کند، همچنان به قوت خود باقی است.

چالش تعمیم‌پذیری به داروهای جدید یکی از محدودیت‌های اصلی روش‌های موجود است \cite{ref_deng2020}. این مسئله به ویژه در مورد داروهایی که هیچ داده‌ای از آن‌ها در مجموعه آموزش وجود ندارد، چالش‌برانگیز است. از سوی دیگر، مقیاس‌پذیری نیز چالش جدی دیگری محسوب می‌شود \cite{ref_dai2020}؛ بسیاری از روش‌های موجود در مواجهه با حجم زیاد داده و تعداد زیاد انواع تداخلات با مشکل مواجه می‌شوند و نیاز به روش‌هایی که بتوانند در مقیاس بزرگ و با کارایی محاسباتی مناسب عمل کنند، به شدت احساس می‌شود. محدودیت دیگر، پوشش ناکافی انواع تداخلات است. بسیاری از مطالعات موجود تنها بر تعداد محدودی از انواع تداخلات تمرکز کرده‌اند \cite{ref_ryu2018} و نیاز به روش‌هایی با قابلیت پوشش طیف گسترده‌تر تداخلات و حفظ دقت قابل قبول در پیش‌بینی هر نوع تداخل، همچنان احساس می‌شود.

علاوه بر موارد ذکر شده، دو چالش مهم دیگر نیز در این حوزه وجود دارد. نخست، مسئله استفاده از داده‌های متنی است؛ علی‌رغم وجود حجم زیاد اطلاعات ارزشمند در متون علمی و پزشکی، اکثر روش‌های موجود از این منبع غنی اطلاعات به خوبی استفاده نمی‌کنند \cite{ref_he2023}. دوم، چالش تفسیرپذیری مدل‌هاست که به عنوان یکی از نیازهای اساسی در کاربردهای بالینی مطرح می‌شود. اگرچه برخی روش‌های موجود دقت قابل قبولی در پیش‌بینی تداخلات دارویی نشان داده‌اند، اما در ارائه توضیحات قابل فهم برای پیش‌بینی‌های خود با محدودیت مواجه هستند \cite{ref_lin2022}. این موضوع به ویژه در محیط‌های بالینی که نیاز به درک دلیل پیش‌بینی‌ها وجود دارد، اهمیت بیشتری پیدا می‌کند.

در این پژوهش، یک رویکرد جامع و چندوجهی برای پوشش برخی از این شکاف‌های تحقیقاتی ارائه شده است. مدل پیشنهادی با بهره‌گیری همزمان از اطلاعات ساختاری، شباهت‌های دارویی و داده‌های متنی، به چالش محدودیت استفاده از منابع مختلف داده می‌پردازد. همچنین، با استفاده از معماری‌های پیشرفته یادگیری عمیق و طراحی سناریوهای مختلف ارزیابی، تلاش می‌شود تا قابلیت تعمیم‌پذیری مدل به داروهای جدید بهبود یابد. جزئیات این رویکرد و نحوه پرداختن به هر یک از این چالش‌ها در فصل‌های بعدی به تفصیل مورد بحث قرار خواهد گرفت.
